\documentclass[20pt, a0paper, landscape]{tikzposter}
\usepackage{amsmath}
\usepackage{lipsum}
\usepackage{amssymb}
\usepackage[skip=1em, indent=0pt]{parskip}
\usepackage{graphicx}
\usepackage{xfrac} % for nice fractions
\usepackage{wrapfig} % for wrapping text around figures

\title{Mathematics and Weather}
\author{Erik, Ernie, Jasper, Rosie}
\date{\today}
\institute{Cambridge Maths School}
\usetheme{Default}

\begin{document}
\maketitle

% This is a draft. I understand that we should maybe cut down on the equations and their explanations.

\begin{columns}
    \column{0.25}
    \block{Objectives}{
        Our primary project objectives were:
        \vspace{0.4em}
        \begin{itemize}
            \item Determine how can we use mathematics to model fluids.
            \item Discover what models are used to model the atmosphere.
            \item To answer why it's difficult to predict the weather accurately.
        \end{itemize}
    }
    \block{Navier Stokes Equations}{
        A Newtonian fluid is one where the stress on the fluid is proportional to the rate of strain. The Navier Stokes equations describe the motion of such fluids. The derivation isn't trivial, but in essence it combines Newton's second law, the conservation of mass, and equates shear stresses to produce the following differential equations:
        \vspace{1em}
        \begin{align}
            \frac{\partial \mathbf{u}}{\partial t} + (\mathbf{u} \cdot \nabla) \mathbf{u} &= -\frac{1}{\rho} \nabla p + \nu \nabla^2 \mathbf{u} + \mathbf{F} \\
            \nabla \cdot \mathbf{u} &= 0
        \end{align} \par
        \vspace{0.75em}
        The left-hand side of (1) is the material derivative of velocity, and describes how velocity changes from the perspective of a particle. The right hand side is the sum of the pressure gradient, viscous (turbulent) forces, and external forces. Equation (2) is the incompressibility condition, which states that matter is not created or destroyed with respect to volume.
        \par
        Because of the $\frac{\text{D}\mathbf{u}}{\text{D} t}$ term, these equations are non-linear and difficult to solve. We are therefore forced to use numerical methods to approximate solutions. \par
        The Navier Stokes equations also assume incompressibility, which as we all know, is not true for gases. However, this has very little effect on the accuracy of the model since variations in density are small and only occur near the speed of sound - nowhere near the speeds of weather systems. \par
        You'll also notice that the equations lack any mention of temperature, which will cause variation in density and pressure.
    }
    \block{Primitive Equations}{
        These equations are a simplification of the Navier Stokes equations, and are used to model the atmosphere. We assume that the atmosphere is a thin layer of fluid, and that the Coriolis force is the only force acting on the fluid. This gives us the following equations:
        \vspace{1em}
        \begin{gather}
            \frac{\text{D}u}{\text{D}t} - f_0v = -\frac{1}{\rho} \frac{\partial p}{\partial x} \\
            \frac{\text{D}v}{\text{D}t} + f_0u = -\frac{1}{\rho} \frac{\partial p}{\partial y} \\
            \frac{\partial p}{\partial z} = -\rho g \\
            \nabla \cdot \textbf{u} = 0 \\
            c_p \frac{\partial T}{\partial t} = Q
        \end{gather} \par
        \vspace{0.75em} 
        Equations (3) and (4) are the Navier Stokes equations in the horizontal plane but with turbulence omitted for simplicity. Equation (5) is the hydrostatic equation and models the weight of the atmosphere. Equation (6) is the incompressibility condition. Equation (7) equates temperature to heat flow into the system.
    }
    \column{0.25}
    \block{Rayleigh-Bénard Convection Model}{
        The RB model considers two plates with fluid between and the lower plate being heated.
        \vspace{1em}
        \begin{gather}
            \frac{\partial \textbf{u}}{\partial t} = \sigma \nabla^2 \textbf{u} - \nabla p + \sigma \text{Ra} T \hat{\textbf{z}} \\
            \frac{\partial T}{\partial t} = w + \nabla^2 T \\
            \nabla \cdot \textbf{u} = 0
        \end{gather}\par
        \vspace{0.75em}
        %TODO: Explain what the Rayleigh and Prandtl numbers are and what the model represents
        \lipsum[2]
    }
    \block{Lorenz Model and Chaos Theory}{
        The final mathematical model we investiagted was the Lorenz model, which is obtained by applying applying the Fourier expansion and Galerkin Truncation to the Rayleigh-Bénard model. We did not investigate the derivation of this model since it was beyond the scope of our project.

        % TODO: explain what x,y,z,sigma,rho,beta are
        \vspace{1em}
        \begin{align}
            \frac{\partial x}{\partial t} &= \sigma(y - x) \\
            \frac{\partial y}{\partial t} &= x(\rho - z) - y \\
            \frac{\partial z}{\partial t} &= xy - \beta z
        \end{align}\par
        \vspace{0.75em}
        We approximated and plotted the solutions using the simple Euler method. \par
        \begin{minipage}{0.1\textwidth}
            \begin{tikzfigure}[Lorenz Attractor, coloured by position.]
                \includegraphics[width=\textwidth]{Lorenz.png}
            \end{tikzfigure}
        \end{minipage}
        \hfill
        \begin{minipage}{0.1\textwidth}
            \begin{tikzfigure}[Lorenz Attractor, coloured by time.]
                \includegraphics[width=\textwidth]{Lorenz2.png}
            \end{tikzfigure}
        \end{minipage}
        \vspace{1em}
        \par
        This shape is where the name "butterfly effect" comes from. Figure 2 shows how the attractor first stays roughly in the left wing, but spirals out and starts oscillating unpredictably between the two sides.\par
        
    }
    \column {0.25}

    % Too many Butterflies?
    \block{}{
        \begin{wrapfigure}{r}{0.5\linewidth}
            \begin{tikzfigure}[Lorenz Attractor, $1\times 10^5$ steps.]
                \includegraphics[width=0.1\textwidth]{Lorenz3.png}
            \end{tikzfigure}
            \begin{tikzfigure}[Lorenz Attractor, $2\times 10^5$ steps.]
                \includegraphics[width=0.1\textwidth]{Lorenz4.png}
            \end{tikzfigure}
            \begin{tikzfigure}[Lorenz Attractor, $3\times 10^5$ steps.]
                \includegraphics[width=0.1\textwidth]{Lorenz5.png}
            \end{tikzfigure}
        \end{wrapfigure}\par
        Figures 3-5 show how slight variations in starting conditions can lead to vastly different outcomes. Each simulation had initial $x,y,z$ following a normal distribution with standard deviation $0.001$. Figure 1 shows a predictable state where all attractors have roughly the same state and this variation has had no noticeable effect. Figure 2 shows some diversion, and figure three shows a chaotic system. Final states are shown by red dots.

        \lipsum[1]
    }
\end{columns}

\end{document}
